
\section {Overview}
\label{sec-overview}

\begin{figure*}[!t]
  \centering \subfloat[NYTimes
  Paywall]{\includegraphics[width=0.48\textwidth]{diagrams/nytimes}}
  \hfil \subfloat[Cookie
  Replay]{\includegraphics[width=0.51\textwidth]{diagrams/replay}}
\caption{Models of (a) the New York Times paywall system and (b)
  a cookie replay attack. A box represents a module, and a
  directed edge from one module \textsf{A} to another module
  \textsf{B} with label \textsf{O} represents \textsf{A}'s invocation
  of operation \textsf{O} that is exported by \textsf{B}.}
\label{fig-nytimes}
\end{figure*}

In this section, we give an informal overview of how an analyst would
use the framework to incrementally construct and analyze a model of a
system. We use an example from the web application domain to
illustrate our appraoch.

\paragraph{\textbf{Building an Initial Model}} Let us assume that the
\textit{New York Times}, a well-known news organization, wishes to
design a \textit{paywall} system for its online web site. A paywall
system is used to limit readers to a set number of free articles over
a time period, requiring them to signup for a paid subscription once
they reach the limit. One desirable property of this paywall
system is that \textit{readers who have already exceeded the limit
  should not be able to access an article}.

Alice, the chief web designer of the New York Times, devises a simple
design of the paywall system, as shown in
Figure~\ref{fig-nytimes}(a). There are three basic participants in this
design: the \textsf{NYTimes} server, which stores and serves articles,
\textsf{Reader}, an end-user who wishes to access an article, and
\textsf{Client}, which mediates the transfer of articles between
\textsf{NYTimes} and \textsf{Client}. When \textsf{Reader} selects a
link to an article, \textsf{Client} forwards the request for the
article to the \textsf{NYTimes} server, sending along a counter
(\textsf{currCounter}) that represents the number of articles
\textsf{Reader} has accessed so far. \textsf{NYTimes} accepts a
request \textsf{GetPage} from \textsf{Client} only if
\textsf{currCounter} is less than the limit. Once processing the
request, \textsf{NYTimes} then sends back a page that contains the
requested article, along with an increment of the original counter
(\textsf{newCounter}).

Note that the model in Figure~\ref{fig-nytimes}(a) describes the system
interactions in terms of web API operations. This level of abstraction
is suitable for the designer to capture the essential functionality of
the system; it omits low-level details that are irrelevant to the
system workflow, such as what kind of devices various participants
are deployed on, how various API operations are implemented, or the
type of data structure that is used to represent counters and pages.

To ensure that the design is satisfactory, Alice decides to run the
model in Figure~\ref{fig-nytimes}(a) through an analysis engine that
exhaustively explores the behavior of the model. As expected, the
engine concludes that the system satisfies the property by preventing
the reader from accessing an article beyond the limit.

\paragraph{\textbf{Elaborating the Model}} Being satisfied with the
initial design, Alice moves onto exploring lower-level design
choices for the system. She decides that \textsf{NYTimes} will run on
top of an HTTP server, with \textsf{GetPage} implemeneted as a
standard HTTP GET operation. Similarly, she
designates \textsf{Client} to run on top of a standard web browser,
and the article counter to be stored as a cookie inside the browser.

Alice decides to explore the potential implications of these design
decisions on the security of the system using our framework. Let us
assume that the repository is already populated with models that
describe different types of attacks that are applicable to web-based
systems. For example, basd on Alice's decisions, one type of attack
that the paywall system might be vulnerable to is a \textit{cookie
  replay attack}, which involves a malicious user obtaining a cookie
and repeating its transmission to a server. In this model, shown in
Figure~\ref{fig-nytimes}(b), \textsf{Browser} communicates to
\textsf{HTTPServer} by sending \textsf{GET} requests and receiving
responses (\textsf{Resp}) in return. In addition, a malicious user
(named \textsf{ReplayAttacker}) interacts with \textsf{Browser} by
extracting a cookie from it (\textsf{ExtractCookie}), or overwriting
an existing cookie at a particualr address (\textsf{SetCookie}); this
allows \textsf{ReplayAttacker} to manipulate \textsf{Browser} into
sending a request with a fixed cookie value.

Note the \textit{abstraction mismatch} between the two model in
Figure~\ref{fig-nytimes}. They describe the system at different layers
of abstraction, using two distinct sets of vocabulary terms to
describe modules, operations, and data elements. These are
\textit{partial} descriptions of a system, including only the details
of the system that are necessary to illustrate a typical workflow (as
in the paywall model) or the characteristics of a particular attack
(in the replay model). Thus, in their current form, they are not
readily amenable to composition, and additional guidance is needed to
identify the relationships between the two models.

Intuitively, relationships between two models correspond to the
designer's decisions about how different parts of the system are to be
realized in a more concrete representation. For example, based on Alice's
decisions about the implementation of the paywall system, 

\begin{figure}[!t]
\centering
\includegraphics[width=0.9\textwidth]{diagrams/merged}
\caption{Result of merging the two models from Figure~\ref{fig-nytimes}.}
\label{fig-merged}
\end{figure}

\paragraph{\textbf{Re-anayzing the Model}}

(describe step by step iterative process)

TBD
