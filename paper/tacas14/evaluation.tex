
\section{Case Study}
\label{sec-case-study}

We demonstrate that our framework can be used to incrementally explore
design alternatives for a web application and analyze their
implications on the security of the system. Types of design decisions
in the web application domain include:
\begin{itemize}
\item Authentication mechanism: Passwords, certificates,
  Kerberos, or more advanced protocols such as OpenID and OAuth.
\item Critical resource storage: Decisions about (1) where to store a
  piece of critical data (e.g., store a user session on the client vs
  server side), and (2) how to store it (in plain text vs encrypted).
\item Data transfer protocols: Mechanisms to transfer a piece of data
  between its components and protect it from malicious users (e.g,
  SSL vs TLS).
\item Deployment architecture: Decisions about the type of device onto
  which a component will be deployed (e.g., browser vs mobile app).
\item Implementation languages: The choice of a programming language
  used to implement a particular component.
\end{itemize}
Each decision may introduce different types of vulnerabilities into
the system. Furthermore, one or more of these decisions interact with
each other in subtle ways.

We have constructed a repository that contains models of these design
alternatives and their associated vulnerabilities. We have applied our
framework to analyze the design of several web applications \ek{what
  are these?}. For each of the applications, we began by building an
intiial model of a system that described the core system functionality
along with a desired security property, but omitted low-level design
decisions. We manually extracted the actual design choices of the
application, and incrementally added them to the model, each time
running our analysis engine to check whether the system might be
vulnerable to a security attack. When the analysis engine generated a
counterexample, we manually confirmed that it indeed represented a
real vulnerability by replaying a concretized version of the
counterexample on the system. In the following section, we report on
the vulnerabilities that we discovered in these applications, and on
the performance of the analysis engine.

