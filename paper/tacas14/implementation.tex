
\section{Implementation}
\label{sec-implementation}

To implement our approach, we (1) designed a domain-specific surface
language (called \sLangLong), and (2) wrote a compiler that translates
models written in \sLang to first-order relational logic (namely the
Alloy language~\cite{alloy}), which is amenable to fully automated
(but bounded) analysis.

The main design goal of \sLang is to provide a high-level paradigm for
writing formal system designs (architectures) which is intuitive and
readily accessible to analysts not already experienced in formal
methods.  \sLang, therefore, intentionally hides most of the internal
complexities of our declarative formalism; instead, \sLang provides a
mostly imperative service-oriented paradigm, which allows the analyst
to express valid system behaviors simply by specifying service
invocations that different components may perform.  Furthermore,
\sLang is fully embedded in Ruby (all \sLang programs are
syntactically correct Ruby programs), making the extensive set of
existing development tools for Ruby readily available, as well as
making the learning curve less steep.

In this paper, we do not attempt to evaluate the assertions about the
language simplicity and accessibility to non-experts in formal
methods; instead, we invite the community to contribute our existing
knowledge database by downloading \sLang~\cite{slang-web} and
submitting attack models in their domain of expertise.\check{!}

\begin{figure}[ht]
\centering
  \begin{slang}
view :Paywall do
  critical data Page
  data Link
  
  trusted component NYTimes [articles: Int ** Page, limit: Int] do
    creates Article
    operation GetPage [link: Link, currCounter: Int] do
      guard { currCounter < limit) }
      sends { Client::SendPage[articles[link]] }
    end
  end

  trusted component Client [counter: Int] do
    operation SendPage[article: Article] do 
      sends { Reader::Display[article] }
    end
    operation SelectLink[link: Link] do
      sends { NYTimes::GetPage[link, counter] }
    end
  end

  component Reader do
    operation Display[page: Page]
    sends { Client::SelectLink }
  end
end
  \end{slang}

\caption{Paywall example in \sc{Slang}}
\label{fig-paywall-slang}
\end{figure}

