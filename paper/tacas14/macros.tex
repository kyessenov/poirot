%% For code that appears on its own line
\newenvironment{CodeOut}{\begin{scriptsize}}{\end{scriptsize}}
%% For code that appears within text. Example: \CodeIn{my_code}
\newcommand{\CodeIn}[1]{\begin{small}\texttt{#1}\end{small}}
\newcommand{\code}[1]{\CodeIn{#1}}
\newcommand{\bcode}[1]{\textbf{\code{#1}}}
\newcommand{\CITE}{$^{[\textcolor{red}{{\bf CITE}}]}$}
\newcommand{\CCITE}[1]{$^{[\textcolor{red}{{\bf CITE(\text{#1})}}]}$}

\def\SEarrow{\ensuremath{\rotatebox[origin=c]{-45}{$\leftrightarrow$}}\xspace}
\def\NEarrow{\ensuremath{\rotatebox[origin=c]{45}{$\leftrightarrow$}}\xspace}

\newcommand{\true}{\CodeIn{true}\xspace}
\newcommand{\false}{\CodeIn{false}\xspace}
\newcommand{\suppress}[1]{}
\newcommand{\suppresss}[2]{}

\renewcommand\algorithmicindent{1.0em}


%%%%%%%%%%%%%%%%%%%%%%%%%%%%%%%%%%%%%%%%%%%%%%%%%%%%
%% ---------- Different font in captions -----------
%%%%%%%%%%%%%%%%%%%%%%%%%%%%%%%%%%%%%%%%%%%%%%%%%%%%

%% \newcommand{\captionfonts}{\normalsize}

%% \makeatletter  % Allow the use of @ in command names
%% \long\def\@makecaption#1#2{%
%%   \vskip\abovecaptionskip
%%   \sbox\@tempboxa{{\captionfonts #1: #2}}%
%%   \ifdim \wd\@tempboxa >\hsize
%%     {\captionfonts #1: #2\par}
%%   \else
%%     \hbox to\hsize{\hfil\box\@tempboxa\hfil}%
%%   \fi
%%   \vskip\belowcaptionskip}
%% \makeatother   % Cancel the effect of \makeatletter

%%%%%%%%%%%%%%%%%%%%%%%%%%%%%%%%%%%%%%%%%%%%%%%%%%%%
%% ------------- conditional switches --------------
%%%%%%%%%%%%%%%%%%%%%%%%%%%%%%%%%%%%%%%%%%%%%%%%%%%%
\def\MREV{reviewing}
\def\MFIN{final}
\def\moderev{\MREV}
\def\modefin{\MFIN}

\def\CMINTED{minted}
\def\CLISTING{listing}
\def\codeminted{\CMINTED}
\def\codelisting{\CLISTING}


\def\aleks#1{\fix{}{#1}}
\def\am#1{\comment{AM}{#1}}
\def\ek#1{\comment{EK}{#1}}


\def\Tiny{\fontsize{4pt}{4pt}\selectfont}

%% ------------------------------------------------
%% when all fixes have been approved, and comments 
%% taken care of, change the previous macros to
%% ------------------------------------------------

% \newcommand{\comment}[2]{}
% \newcommand{\todo}[1]{} 
% \newcommand{\del}[1]{}
% \newcommand{\fix}[2]{#2}

% Lyle Ramshaw's unbreakable hyphen
\newcommand{\unbreakableHyphen}{\setbox0=\hbox{-}\setbox1=\hbox{-\/}%
        \kern\wd0\kern-\wd1\hbox{-}}

\newcommand{\secref}[1]{Section~\ref{#1}}  % for use in text
\newcommand{\Secref}[1]{Section~\ref{#1}}  % for start of sentence
\newcommand{\secrefs}[2]{Section~\ref{#1} and~\ref{#2}}  % for use in text
\newcommand{\figref}[1]{Figure~\ref{#1}}     % for use in text
\newcommand{\Figref}[1]{Figure~\ref{#1}}   % for start of sentence
\newcommand{\Figrefs}[2]{Figures~\ref{#1} and~\ref{#2}}
\newcommand{\figrefs}[2]{Figures~\ref{#1} and~\ref{#2}}

